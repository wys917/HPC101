\subsection{Point-to-Point Communication}

\begin{frame}[fragile]{Blocking Send and Receive}
\begin{columns}
    
\begin{column}{.42\textwidth}
\begin{tabular}{c}
      \begin{minted}[fontsize=\footnotesize]{c}
int MPI_Send(
    const void* buffer,
    int count,
    MPI_Datatype datatype,
    int recipient,
    int tag,
    MPI_Comm communicator);    
      \end{minted}
\end{tabular}
\end{column}

\begin{column}{.58\textwidth}
\textbf{\large Parameters:}

\begin{itemize}
    \item \textbf{buffer} The buffer to send.

    \item \textbf{count} The number of elements to send.

    \item \textbf{datatype} The type of one buffer element.

    \item \textbf{recipient} The rank of the recipient MPI process.

    \item \textbf{tag} The tag to assign to the message.

    \item \textbf{communicator} The communicator in which the standard send takes place.
\end{itemize}

\end{column}
\end{columns}
\end{frame}

\begin{frame}[fragile]{Blocking Send and Receive}
\begin{columns}
    
\begin{column}{.4\textwidth}
\begin{tabular}{c}
    \begin{minted}[fontsize=\footnotesize]{c}
int MPI_Recv(
    void* buffer,
    int count,
    MPI_Datatype datatype,
    int sender,
    int tag,
    MPI_Comm communicator,
    MPI_Status* status);
      \end{minted}
\end{tabular}
\end{column}

\begin{column}{.6\textwidth}
\textbf{\large Parameters:}
{\footnotesize
\begin{itemize}
    \item \textbf{buffer} The buffer to receive.

    \item \textbf{count} The number of elements to receive.

    \item \textbf{datatype} The type of one buffer element.

    \item \textbf{sender} The rank of the sender MPI process.

    \item \textbf{tag} The tag to assign to the message.

    \item \textbf{communicator} The communicator in which the standard receive takes place.

    \item \textbf{status} The variable in which store the status of the receive operation. Pass MPI\_STATUS\_IGNORE if unused.
\end{itemize}
}
\end{column}
\end{columns}
\end{frame}

\begin{frame}{MPI\_Status}
    MPI\_Status represents the status of a reception operation.

    At least 3 attributes: 
    \begin{itemize}
        \item MPI\_SOURCE
        \item MPI\_TAG
        \item MPI\_ERROR
    \end{itemize}

    There may be additional attributes that are implementation-specific.
\end{frame}

\begin{frame}{Message Envelope}
    In addition to the data part, messages carry information that can be used to distinguish messages and selectively receive them.
    \begin{itemize}
        \item source
        \item destination
        \item tag
        \item communicator
    \end{itemize}
\end{frame}


\begin{frame}{Communication Mode}
\only<1>{
    \begin{itemize}
        \item \textbf{Buffer Mode}
        
        Can be started whether or not a matching receive was posted.
        Completion does not depend on the occurrence of a matching receive.
        \item \textbf{Synchronous Mode}

        Can be started whether or not a matching receive was posted.

        The send will be completed successfully only if a matching receive is posted.

        \item \textbf{Ready Mode}

        May be started only if the matching receive is already posted.

        \item \textbf{Standard Mode}

        Depends.
        
    \end{itemize}
}

\only<2>{
\footnotesize

    \begin{table}[]
\begin{tabular}{lll}

\textbf{Communication mode} & \textbf{Start time}   & \textbf{Completion time}                         \\
\hline
Buffer mode        & Immediately                 & Message has gone to buffer              \\
\hline
Synchronous mode   & Immediately                 & Matching receive has posted \\
\hline
Ready mode         & Matching receive has posted & When the send buffer can be reused      \\
\hline
Standard mode      & Depends                     & Depends  \\                  \hline      

\end{tabular}


\end{table}
}
\end{frame}

\begin{frame}[fragile]{Communication mode: A common mistake}

    \textbf{Note}: MPI\_Ssend will \textbf{always wait until the receive has been posted} on the receiving end.
    \begin{minted}[fontsize=\footnotesize]{c}
// n = 2
MPI_Comm_rank(comm, &my_rank);
MPI_Ssend(sendbuf, count, MPI_INT, my_rank ^ 1, tag, comm);
MPI_Recv(recvbuf, count, MPI_INT, my_rank ^ 1, tag, comm, &status);
      \end{minted}

    \onslide<1->{\emoji{thinking} What will happen?}

    \onslide<2>{\emoji{exploding-head} Deadlock! Any solutions?}
\end{frame}

\begin{frame}[fragile]{Communication mode: Solution}
    \begin{tabular}{c}
      \begin{minted}[fontsize=\footnotesize]{c}
// n = 2
MPI_Comm_rank(comm, &my_rank);
if (my_rank == 0) {
    MPI_Ssend(sendbuf, count, MPI_INT, 1, tag, comm);
    MPI_Recv(recvbuf, count, MPI_INT, 1, tag, comm, &status);
} else if (my_rank == 1) {
    MPI_Recv(recvbuf, count, MPI_INT, 0, tag, comm, &status);
    MPI_Ssend(sendbuf, count, MPI_INT, 0, tag, comm);
}
      \end{minted}
    \end{tabular}

    \onslide<2->{\emoji{thinking} Any other solutions?}
\end{frame}

\begin{frame}[fragile]{Blocking Send and Receive}
\begin{columns}
    
\begin{column}{.5\textwidth}
\begin{minted}[fontsize=\footnotesize]{c}
int MPI_Sendrecv(
    const void* buffer_send,      
    int count_send,
    MPI_Datatype datatype_send,
    int recipient,
    int tag_send,
    void* buffer_recv,
    int count_recv,
    MPI_Datatype datatype_recv,
    int sender,
    int tag_recv,
    MPI_Comm communicator,
    MPI_Status* status);
\end{minted}
\end{column}

\begin{column}{.5\textwidth}
\begin{block}{Notice}
    The buffers used for send and receive must be different.
\end{block}

\onslide<2->{\emoji{thinking} Any other solutions?}

\end{column}
\end{columns}
\end{frame}

\begin{frame}[fragile]{Non-Blocking Send and Receive}
\begin{block}{Recall}
    A nonblocking call initiates the operation,
    but does not complete it.
    
    They will return almost immediately.
\end{block}
    
\begin{minted}[fontsize=\footnotesize]{c}
int MPI_Isend(const void* buffer,
              int count,
              MPI_Datatype datatype,
              int recipient,
              int tag,
              MPI_Comm communicator,
              MPI_Request* request);
\end{minted}
\end{frame}

\begin{frame}{Synchronization}
    \begin{columns}
        \begin{column}{.5\textwidth}
        
            \begin{itemize}
                \item \textbf{MPI\_Test}

                MPI\_TEST(request, flag, status)
                
                \begin{itemize}

                \item Checks if a non-blocking operation is complete at a given time.
                
                \item flag=true if completes.
                \end{itemize}
            \end{itemize}
        \end{column}

        \begin{column}{.5\textwidth}
            \begin{itemize}
                \item \textbf{MPI\_Wait}

                MPI\_WAIT(request, status)

                \begin{itemize}
                    \item Waits for a non-blocking operation to complete. 
                    \item That is, unlike MPI\_Test, MPI\_Wait will block until the underlying non-blocking operation completes.
                \end{itemize}
            \end{itemize}
        \end{column}
    \end{columns}
\end{frame}

\begin{frame}[fragile]{Non-Blocking Send and Receive(Deadlock revisit)}
\begin{minted}[fontsize=\footnotesize]{c}
MPI_Request req;
MPI_Isend(sendbuf, 0x100, MPI_INT, my_rank^1, 0, MPI_COMM_WORLD, &req);
MPI_Recv(recvbuf, 0x100, MPI_INT, my_rank^1, 0, MPI_COMM_WORLD, MPI_STATUS_IGNORE);
\end{minted}
\end{frame}